\documentclass[12pt,a4paper]{article}
\usepackage[UTF8]{ctex}         % 中文支持
\usepackage{amsmath, amssymb}   % 数学公式
\usepackage{geometry}           % 页面设置
\usepackage{hyperref}           % 超链接
\usepackage{graphicx}           % 插图
\usepackage{enumerate}          % 枚举
\usepackage{booktabs}           % 表格
\usepackage{lmodern}            % 现代字体
\usepackage{microtype}          % 微排版优化
\usepackage{xcolor}             % 颜色支持
\usepackage[most]{tcolorbox}   % 彩色盒子
\geometry{left=2cm, right=2cm, top=2cm, bottom=2.5cm}

\definecolor{ProblemFrame}{HTML}{2F5597}
\definecolor{ProblemBack}{HTML}{EEF2FB}
\definecolor{SolutionColor}{HTML}{007060}

\hypersetup{
	colorlinks=true,
	linkcolor=ProblemFrame,
	citecolor=SolutionColor,
	urlcolor=ProblemFrame
}

% --- 大标题样式调整 ---
\makeatletter
\def\@maketitle{%
  \newpage
  \null
  \vspace{-2em} % 减少顶部空白
  \begin{center}%
  \let \footnote \thanks
    {\LARGE \@title \par}%
    \vskip 1.5em%
    {\large
      \lineskip .5em%
      \begin{tabular}[t]{c}%
        \@author
      \end{tabular}\par}%
    \vskip 1em%
    {\large \@date}%
  \end{center}%
  \par
  \vskip 1.5em}
\makeatother


% --- 新的题解环境 ---
\newtcolorbox{exercisebox}[2][]{
    enhanced,
    breakable,
    sharp corners,
    colback=white,
    colframe=black!15!white,
    coltitle=white,
    fonttitle=\bfseries\large,
    attach boxed title to top left={xshift=1.5mm, yshift*=-\tcboxedtitleheight/2},
    boxed title style={colback=ProblemFrame, sharp corners},
    title={#2},
    before skip=18pt,
    after skip=18pt,
    #1
}

\newtcolorbox{problemnobox}[1][]{
    enhanced,
    sharp corners,
    boxrule=0.8pt,
    colframe=black!25!white,
    colback=ProblemBack,
    breakable,
    top=4pt, bottom=4pt, left=5pt, right=5pt,
    #1
}

\newenvironment{exercise}[2][]
  {\begin{exercisebox}[#1]{#2}\ignorespaces}
  {\end{exercisebox}}

\newenvironment{problem}
  {\begin{problemnobox}\ignorespaces}
  {\end{problemnobox}}

\newenvironment{solution}
  {\par\noindent\textbf{\color{SolutionColor}解:}\ignorespaces}
  {}

\newenvironment{proof}
  {\par\noindent\textbf{\color{SolutionColor}证明:}\ignorespaces}
  {}

\setlength{\parskip}{0.85em}
\setlength{\parindent}{2em}

\title{数据科学的数学基础 \\ 第{{6}}次理论作业 }
\author{ 陈万祺 \quad 3220102895}
\date{\today}

\begin{document}

\maketitle

\begin{exercise}{Problem 1}
\begin{problem}
Given the following two problems:
\[
\min_{x\in\mathbb{R}^{n}} \|Ax-b\|_{2}^{2} + s \|x\|_{2}^{2} \quad \text{(1)}
\]
and
\[
\min_{x\in\mathbb{R}^{m}, \|x\|_{2}^{2} \leq t} \|Ax-b\|_{2}^{2} \quad \text{(2)}.
\]
Show that the two problems are equivalent in the sense that for any \(s>0\), there exists \(t>0\), such that the solution of problem (1) is also the solution of problem (2).
\end{problem}
\begin{solution}
设 \(x^*\) 为问题 (1) 的最优解。由一阶必要条件可得
\[
\nabla \bigl(\|Ax - b\|_{2}^{2} + s \|x\|_{2}^{2}\bigr)\big|_{x=x^*}
  = 2A^\top (Ax^* - b) + 2s x^* = 0,
\]
从而
\[
(A^\top A + sI)x^* = A^\top b.
\]
令 \(t = \|x^*\|_{2}^{2}\),则 \(x^*\) 满足问题 (2) 的约束。为证 \(x^*\) 同时最优,假设存在 \(x'\) 满足 \(\|x'\|_{2}^{2} \le t\) 且 \(\|Ax' - b\|_{2}^{2} < \|Ax^* - b\|_{2}^{2}\)。

利用 \(\|x'\|_{2}^{2} \le t\) 得
\[
\|Ax' - b\|_{2}^{2} + s\|x'\|_{2}^{2}
  \le \|Ax' - b\|_{2}^{2} + s t
  < \|Ax^* - b\|_{2}^{2} + s t
  = \|Ax^* - b\|_{2}^{2} + s\|x^*\|_{2}^{2},
\]
这与 \(x^*\) 是问题 (1) 的最优解矛盾。因此,对任意 \(s>0\) 取 \(t = \|x^*\|_{2}^{2}\),问题 (1) 与问题 (2) 的解相同。
\end{solution}
\end{exercise}

\begin{exercise}{Problem 2}
\begin{problem}
Apply the matching pursuit algorithm to solve the LASSO problem (only do two iterations):
\[
\min_{x\in\mathbb{R}^{3}} \|Ax-b\|_{2}^{2} + s \|x\|_{1} \quad \text{(3)},
\]
with \(s=0.5\),
\[
A=\begin{bmatrix}2 & 1 & -2 \\ 1 & -1 & 1\end{bmatrix}
\]
and
\[
b=\begin{bmatrix}1 \\ 3\end{bmatrix}.
\]
\end{problem}
\begin{solution}
考虑 LASSO 目标
\[
\min_{x\in\mathbb{R}^3} \|Ax-b\|_2^2 + s\|x\|_1,
\quad s = 0.5,
\quad
A = \begin{bmatrix}2 & 1 & -2\\ 1 & -1 & 1\end{bmatrix},
\quad
b = \begin{bmatrix}1\\3\end{bmatrix}.
\]
记各列为 \(A_1, A_2, A_3\)。匹配追踪开启前,计算列范数
\[
\|A_1\|_2^2 = 5,\qquad
\|A_2\|_2^2 = 2,\qquad
\|A_3\|_2^2 = 5,
\]
以及软阈值参数
\[
\lambda_j = \frac{s}{2\|A_j\|_2^2}
\quad (j=1,2,3)
\Rightarrow
\lambda_1 = 0.05,\; \lambda_2 = 0.125,\; \lambda_3 = 0.05.
\]

初始化 \(x^{(0)} = 0\),残差 \(r^{(0)} = b\)。以下给出两轮迭代:

1. 相关性
  \[
  c = A^\top r^{(0)} = \begin{bmatrix}5\\-2\\1\end{bmatrix}.
  \]
  取绝对值最大的 \(c_1\),得到临时解
  \[
  a = \frac{c_1}{\|A_1\|_2^2} = 1,
  \qquad
  x_1^{(1)} = S_{\lambda_1}(a) = 0.95,
  \]
  其中软阈值算子 \(S_{\lambda}(a) = \operatorname{sign}(a)\max(|a|-\lambda,0)\)。更新向量与残差
  \[
  x^{(1)} = \begin{bmatrix}0.95\\0\\0\end{bmatrix},
  \qquad
  r^{(1)} = r^{(0)} - A_1(x_1^{(1)} - 0)
        = \begin{bmatrix}-0.90\\2.05\end{bmatrix}.
  \]

2. 再次计算相关性
  \[
  c = A^\top r^{(1)} = \begin{bmatrix}0.25\\-2.95\\3.85\end{bmatrix},
  \]
  最大分量对应列 \(A_3\)。于是
  \[
  a = \frac{c_3}{\|A_3\|_2^2} = 0.77,
  \qquad
  x_3^{(2)} = S_{\lambda_3}(a) = 0.72,
  \]
  并更新
  \[
  x^{(2)} = \begin{bmatrix}0.95\\0\\0.72\end{bmatrix},
  \qquad
  r^{(2)} = r^{(1)} - A_3(x_3^{(2)} - 0)
        = \begin{bmatrix}0.54\\1.33\end{bmatrix}.
  \]

完成两次迭代后,匹配追踪得到的系数向量为
\[
x^{(2)} = [0.95,\; 0,\; 0.72]^\top.
\]
\end{solution}
\end{exercise}


\end{document}