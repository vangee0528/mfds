\documentclass[12pt,a4paper]{article}
\usepackage[UTF8]{ctex}         % 中文支持
\usepackage{amsmath, amssymb}   % 数学公式
\usepackage{geometry}           % 页面设置
\usepackage{hyperref}           % 超链接
\usepackage{graphicx}           % 插图
\usepackage{enumerate}          % 枚举
\usepackage{booktabs}           % 表格
\usepackage{lmodern}            % 现代字体
\usepackage{microtype}          % 微排版优化
\usepackage{xcolor}             % 颜色支持
\usepackage[most]{tcolorbox}   % 彩色盒子
\geometry{left=2cm, right=2cm, top=2cm, bottom=2.5cm}

\definecolor{ProblemFrame}{HTML}{2F5597}
\definecolor{ProblemBack}{HTML}{EEF2FB}
\definecolor{SolutionColor}{HTML}{007060}

\hypersetup{
	colorlinks=true,
	linkcolor=ProblemFrame,
	citecolor=SolutionColor,
	urlcolor=ProblemFrame
}

% --- 大标题样式调整 ---
\makeatletter
\def\@maketitle{%
  \newpage
  \null
  \vspace{-2em} % 减少顶部空白
  \begin{center}%
  \let \footnote \thanks
    {\LARGE \@title \par}%
    \vskip 1.5em%
    {\large
      \lineskip .5em%
      \begin{tabular}[t]{c}%
        \@author
      \end{tabular}\par}%
    \vskip 1em%
    {\large \@date}%
  \end{center}%
  \par
  \vskip 1.5em}
\makeatother


% --- 新的题解环境 ---
\newtcolorbox{exercisebox}[2][]{
    enhanced,
    breakable,
    sharp corners,
    colback=white,
    colframe=black!15!white,
    coltitle=white,
    fonttitle=\bfseries\large,
    attach boxed title to top left={xshift=1.5mm, yshift*=-\tcboxedtitleheight/2},
    boxed title style={colback=ProblemFrame, sharp corners},
    title={#2},
    before skip=18pt,
    after skip=18pt,
    #1
}

\newtcolorbox{problemnobox}[1][]{
    enhanced,
    sharp corners,
    boxrule=0.8pt,
    colframe=black!25!white,
    colback=ProblemBack,
    breakable,
    top=4pt, bottom=4pt, left=5pt, right=5pt,
    #1
}

\newenvironment{exercise}[2][]
  {\begin{exercisebox}[#1]{#2}\ignorespaces}
  {\end{exercisebox}}

\newenvironment{problem}
  {\begin{problemnobox}\ignorespaces}
  {\end{problemnobox}}

\newenvironment{solution}
  {\par\noindent\textbf{\color{SolutionColor}解:}\ignorespaces}
  {}

\newenvironment{proof}
  {\par\noindent\textbf{\color{SolutionColor}证明:}\ignorespaces}
  {}

\setlength{\parskip}{0.85em}
\setlength{\parindent}{2em}

\title{数据科学的数学基础 \\ 第{{}}次理论作业 }
\author{ 陈万祺 \quad 3220102895}
\date{\today}

\begin{document}

\maketitle

\begin{exercise}{Problem 1}
\begin{problem}
Prove that the distance induced by Rogers-Tanimoto similarity is a metric.
\end{problem}
\begin{solution}

由定义 Rogers-Tanimoto 相似度为
\[
s_{RT}(p,q) = \frac{n - d_H(p,q)}{n + d_H(p,q)},
\]
诱导距离定义为
\[
d_{RT}(p,q) = \sqrt{ s_{RT}(p,p) + s_{RT}(q,q) - 2 s_{RT}(p,q) }.
\]
由于 \(s_{RT}(p,p) = 1\) 和 \(s_{RT}(q,q) = 1\),我们有
\[
d_{RT}(p,q) = \sqrt{2 (1 - s_{RT}(p,q))} = \sqrt{2 \left(1 - \frac{n - d_H(p,q)}{n + d_H(p,q)} \right)} = \sqrt{2 \left( \frac{2 d_H(p,q)}{n + d_H(p,q)} \right)} = \sqrt{ \frac{4 d_H(p,q)}{n + d_H(p,q)} }.
\]
令 \(h = d_H(p,q)\),则 \(d_{RT}(p,q) = 2 \sqrt{ \frac{h}{n + h} }\)。

现在证明 \(d_{RT}\) 满足度量性质:

1. 非负性:由于 \(h \geq 0\) 且 \(n + h > 0\),所以 \(d_{RT}(p,q) \geq 0\)。

2. 同一性:如果 \(p = q\),则 \(h = 0\),所以 \(d_{RT}(p,q) = 0\)。如果 \(d_{RT}(p,q) = 0\),则 \(\frac{4h}{n+h} = 0\),所以 \(h = 0\),即 \(p = q\)。

3. 对称性:由于 \(d_H(p,q) = d_H(q,p)\),所以 \(d_{RT}(p,q) = d_{RT}(q,p)\)。

4. 三角不等式:考虑函数 \(f(x) = \sqrt{ \frac{x}{n + x} }\),则 \(d_{RT}(p,q) = 2 f(h(p,q))\)。由于 \(d_H\) 是度量,有 \(h(p,r) \leq h(p,q) + h(q,r)\)。函数 \(f(x)\) 是凹函数且 \(f(0) = 0\),因此对于凹函数有 \(f(a + b) \leq f(a) + f(b)\)。于是,
   \[
   f(h(p,r)) \leq f(h(p,q) + h(q,r)) \leq f(h(p,q)) + f(h(q,r)),
   \]
   所以
   \[
   d_{RT}(p,r) = 2 f(h(p,r)) \leq 2 f(h(p,q)) + 2 f(h(q,r)) = d_{RT}(p,q) + d_{RT}(q,r).
   \]
   三角不等式成立。

因此,\(d_{RT}\) 是度量。

\end{solution}
\end{exercise}

\begin{exercise}{Problem 2}
\begin{problem}
Show that the distance induced by Sorensen-Dice similarity is not a metric.
\end{problem}
\begin{solution}

由定义 Sorensen-Dice 相似度为
\[
s_{SD}(p,q) = \frac{2 p \cdot q}{\|p\|_1 + \|q\|_1},
\]

诱导距离定义为
\[
d_{SD}(p,q) = \sqrt{ s_{SD}(p,p) + s_{SD}(q,q) - 2 s_{SD}(p,q) }.
\]

由于 \(s_{SD}(p,p) = 1\) 和 \(s_{SD}(q,q) = 1\),我们有
\[
d_{SD}(p,q) = \sqrt{2 (1 - s_{SD}(p,q))} = \sqrt{2 \left(1 - \frac{2 p \cdot q}{\|p\|_1 + \|q\|_1} \right)} = \sqrt{2 \frac{\|p\|_1 + \|q\|_1 - 2 p \cdot q}{\|p\|_1 + \|q\|_1}}.
\]
注意 \(\|p\|_1 + \|q\|_1 - 2 p \cdot q = |A \Delta B| = d_H(p,q)\),其中 \(A\) 和 \(B\) 是对应的集合。所以
\[
d_{SD}(p,q) = \sqrt{2 \frac{d_H(p,q)}{\|p\|_1 + \|q\|_1}}.
\]

现在,我们通过反例证明 \(d_{SD}\) 不满足三角不等式。考虑三个集合:\(A = \{1\}\),\(B = \{1,2\}\),\(C = \{2\}\),对应的二进制向量为 \(p, q, r\)。假设全集为 \(\{1,2\}\),则 \(n=2\),但这里我们直接计算相似度。

\(|A \cap B| = 1\),\(|A| = 1\),\(|B| = 2\),所以 \(s_{SD}(A,B) = \frac{2 \times 1}{1 + 2} = \frac{2}{3}\),于是 \(d_{SD}(A,B) = \sqrt{2 \left(1 - \frac{2}{3}\right)} = \sqrt{\frac{2}{3}} \approx 0.816\)。

\(|B \cap C| = 1\),\(|B| = 2\),\(|C| = 1\),所以 \(s_{SD}(B,C) = \frac{2 \times 1}{2 + 1} = \frac{2}{3}\),于是 \(d_{SD}(B,C) = \sqrt{\frac{2}{3}} \approx 0.816\)。

\(|A \cap C| = 0\),\(|A| = 1\),\(|C| = 1\),所以 \(s_{SD}(A,C) = \frac{2 \times 0}{1 + 1} = 0\),于是 \(d_{SD}(A,C) = \sqrt{2 (1 - 0)} = \sqrt{2} \approx 1.414\)。

现在检查三角不等式:
\[
d_{SD}(A,B) + d_{SD}(B,C) \approx 0.816 + 0.816 = 1.632 < 1.414 \approx d_{SD}(A,C),
\]
违反三角不等式。因此,\(d_{SD}\) 不是度量。

\end{solution}
\end{exercise}

\end{document}