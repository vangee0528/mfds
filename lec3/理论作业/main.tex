\documentclass[12pt,a4paper]{article}
\usepackage[UTF8]{ctex}         % 中文支持
\usepackage{amsmath, amssymb}   % 数学公式
\usepackage{geometry}           % 页面设置
\usepackage{hyperref}           % 超链接
\usepackage{graphicx}           % 插图
\usepackage{enumerate}          % 枚举
\usepackage{booktabs}           % 表格
\usepackage{lmodern}            % 现代字体
\usepackage{microtype}          % 微排版优化
\usepackage{xcolor}             % 颜色支持
\usepackage[most]{tcolorbox}   % 彩色盒子
\geometry{left=2cm, right=2cm, top=2cm, bottom=2.5cm}

\definecolor{ProblemFrame}{HTML}{2F5597}
\definecolor{ProblemBack}{HTML}{EEF2FB}
\definecolor{SolutionColor}{HTML}{007060}

\hypersetup{
	colorlinks=true,
	linkcolor=ProblemFrame,
	citecolor=SolutionColor,
	urlcolor=ProblemFrame
}

% --- 大标题样式调整 ---
\makeatletter
\def\@maketitle{%
  \newpage
  \null
  \vspace{-2em} % 减少顶部空白
  \begin{center}%
  \let \footnote \thanks
    {\LARGE \@title \par}%
    \vskip 1.5em%
    {\large
      \lineskip .5em%
      \begin{tabular}[t]{c}%
        \@author
      \end{tabular}\par}%
    \vskip 1em%
    {\large \@date}%
  \end{center}%
  \par
  \vskip 1.5em}
\makeatother


% --- 新的题解环境 ---
\newtcolorbox{exercisebox}[2][]{
    enhanced,
    breakable,
    sharp corners,
    colback=white,
    colframe=black!15!white,
    coltitle=white,
    fonttitle=\bfseries\large,
    attach boxed title to top left={xshift=1.5mm, yshift*=-\tcboxedtitleheight/2},
    boxed title style={colback=ProblemFrame, sharp corners},
    title={#2},
    before skip=18pt,
    after skip=18pt,
    #1
}

\newtcolorbox{problemnobox}[1][]{
    enhanced,
    sharp corners,
    boxrule=0.8pt,
    colframe=black!25!white,
    colback=ProblemBack,
    breakable,
    top=4pt, bottom=4pt, left=5pt, right=5pt,
    #1
}

\newenvironment{exercise}[2][]
  {\begin{exercisebox}[#1]{#2}\ignorespaces}
  {\end{exercisebox}}

\newenvironment{problem}
  {\begin{problemnobox}\ignorespaces}
  {\end{problemnobox}}

\newenvironment{solution}
  {\par\noindent\textbf{\color{SolutionColor}解:}\ignorespaces}
  {}

\newenvironment{proof}
  {\par\noindent\textbf{\color{SolutionColor}证明:}\ignorespaces}
  {}

\setlength{\parskip}{0.85em}
\setlength{\parindent}{2em}

\title{数据科学的数学基础 \\ 第{{2}}次理论作业 }
\author{ 陈万祺 \quad 3220102895}
\date{\today}

\begin{document}

\maketitle

\begin{exercise}{Problem 2.2}
\begin{problem}
  Let \(X\) be a random variable that you know is in the range \([-1,2]\) and you know has an expected value of \(E[X]=0\). Use the Markov inequality to upper-bound \(Pr[X>1.5]\).  
(Hint: you will need to use a change of variables.)
\end{problem}
\begin{solution}

由于马尔可夫不等式要求随机变量非负,令 \(Y = X + 1\),则:
\[
Y \in [0, 3], \quad E[Y] = E[X + 1] = E[X] + 1 = 1
\]

注意到:
\[
Pr[X > 1.5] = Pr[Y > 2.5]
\]

对非负随机变量 \(Y\) 应用马尔可夫不等式:
\[
Pr[Y \geq 2.5] \leq \frac{1}{2.5} = \frac{2}{5}
\]
由于 \(Pr[Y > 2.5] \leq Pr[Y \geq 2.5]\),因此:
\[
Pr[X > 1.5] = Pr[Y > 2.5] \leq \frac{2}{5}
\]

\end{solution}
\end{exercise}

\begin{exercise}{Problem 2.4}
\begin{problem}
2.4 Consider a pdf \(f\) so that a random variable \(X\sim f\) has expected value \(E[X]=5\) and variance \(Var[X]=100\). Now consider \(n=16\) iid random variables \(X_1,X_2,\dots ,X_{16}\) drawn from \(f\). Let \(\bar X=\frac1{16}\sum_{i=1}^{16}X_i\).

1. What is \(E[\bar X]\)?

2. What is \(Var[\bar X]\)?  
Assume we know that \(X\) is never smaller than \(0\) and never larger than \(20\).

3. Use the Markov inequality to upper-bound \(Pr[\bar X>8]\).

4. Use the Chebyshev inequality to upper-bound \(Pr[\bar X>8]\).

5. Use the Chernoff–Hoeffding inequality to upper-bound \(Pr[\bar X>8]\).

6. If we increase \(n\) to \(100\), how will the above three bounds be affected?

\end{problem}
\begin{solution}

1. 由于 \(X_1, X_2, \dots, X_{16}\) 是独立同分布随机变量,且 \(E[X] = 5\),则:
\[
E[\bar X] = E\left[\frac{1}{16}\sum_{i=1}^{16} X_i\right] = \frac{1}{16} \sum_{i=1}^{16} E[X_i] = \frac{1}{16} \cdot 16 \cdot 5 = 5.
\]
因此,\(E[\bar X] = 5\).

2. 由于方差性质,且 \(Var[X] = 100\),则:
\[
Var[\bar X] = Var\left[\frac{1}{16}\sum_{i=1}^{16} X_i\right] = \frac{1}{16^2} \sum_{i=1}^{16} Var[X_i] = \frac{1}{256} \cdot 16 \cdot 100 = \frac{1600}{256} = 6.25.
\]
因此,\(Var[\bar X] = 6.25\).

3. 使用马尔可夫不等式。由于 \(X \in [0,20]\),故 \(\bar X \geq 0\)。马尔可夫不等式给出:
\[
Pr[\bar X > 8] \leq Pr[\bar X \geq 8] \leq \frac{5}{8}.
\]

4. 使用切比雪夫不等式。切比雪夫不等式给出:
\[
Pr[|\bar X - E[\bar X]| \geq k] \leq \frac{Var[\bar X]}{k^2}.
\]
这里 \(E[\bar X] = 5\),欲求 \(Pr[\bar X > 8] = Pr[\bar X - 5 > 3]\)。注意到:
\[
Pr[\bar X > 8] \leq Pr[|\bar X - 5| \geq 3] \leq \frac{Var[\bar X]}{3^2} = \frac{6.25}{9} = \frac{25}{36}.
\]

5. 使用 Chernoff–Hoeffding 不等式。由于 \(X_i \in [0,20]\),则 \(\bar X\) 是 \(n=16\) 个独立随机变量的均值,且 \(E[\bar X] = 5\)。Chernoff–Hoeffding 不等式给出:
\[
Pr[\bar X - E[\bar X] \geq t] \leq \exp\left(-\frac{2n t^2}{(b-a)^2}\right),
\]
其中 \(a=0\), \(b=20\),故 \(b-a=20\)。取 \(t = 8 - 5 = 3\),得:
\[
Pr[\bar X > 8] = Pr[\bar X - 5 \geq 3] \leq \exp\left(-\frac{2 \cdot 16 \cdot 3^2}{20^2}\right) = \exp\left(-\frac{288}{400}\right) = \exp(-0.72).
\]

6. 当 \(n\) 增加到 100 时:

由于\(E[\bar X] = 5\) 不变,故马尔可夫界仍为 \(\frac{5}{8}\)。

由于\(Var[\bar X] = \frac{100}{100} = 1\),故切比雪夫界变为 \(\frac{1}{3^2} = \frac{1}{9}\),比原来的更小。

由于指数变为 \(\exp\left(-\frac{2 \cdot 100 \cdot 3^2}{20^2}\right) = \exp(-4.5)\),Chernoff–Hoeffding 界比原来更小。

因此,马尔可夫界不变,切比雪夫界和 Chernoff–Hoeffding 界均减小。

\end{solution}
\end{exercise}

\begin{exercise}{Problem 2.5}
\begin{problem}
Consider a (parked) self-driving car that returns \(n\) iid estimates to the distance of a tree. We will model these \(n\) estimates as a set of \(n\) scalar random variables \(X_1,X_2,\dots ,X_n\) taken iid from an unknown pdf \(f\), which we assume models the true distance plus unbiased noise (the sensor can take many iid estimates in rapid-fire fashion). The sensor is programmed to only return values between \(0\) and \(20\) feet, and that the variance of the sensing noise is \(64\) feet squared. Let \(\bar X=\frac1n\sum_{i=1}^n X_i\). We want to understand as a function of \(n\) how close \(\bar X\) is to \(\mu\), which is the true distance to the tree.

1. Use Chebyshev's inequality to determine a value \(n\) so that \(Pr[|\bar X-\mu|\ge 1]\le 0.5\).

2. Use Chebyshev's inequality to determine a value \(n\) so that \(Pr[|\bar X-\mu|\ge 0.1]\le 0.1\).

3. Use the Chernoff–Hoeffding bound to determine a value \(n\) so that \(Pr[|\bar X-\mu|\ge 1]\le 0.5\).

4. Use the Chernoff–Hoeffding bound to determine a value \(n\) so that \(Pr[|\bar X-\mu|\ge 0.1]\le 0.1\).
\end{problem}

\begin{solution}

1.   
由于 \(X_i\) 独立同分布,且 \(E[X_i] = \mu\),\(Var[X_i] = 64\),则  
\[
Var[\bar X] = \frac{64}{n}.
\]  
切比雪夫不等式给出:  
\[
Pr[|\bar X - \mu| \ge \varepsilon] \le \frac{Var[\bar X]}{\varepsilon^2} = \frac{64}{n \varepsilon^2}.
\]  
令 \(\varepsilon = 1\),要求 \(Pr[|\bar X - \mu| \ge 1] \le 0.5\),即  
\[
\frac{64}{n} \le 0.5 \quad \Rightarrow \quad n \ge 128.
\]  
因此,取 \(n = 128\)。

2.   
令 \(\varepsilon = 0.1\),要求 \(Pr[|\bar X - \mu| \ge 0.1] \le 0.1\),即  
\[
\frac{64}{n \cdot (0.1)^2} \le 0.1 \quad \Rightarrow \quad \frac{6400}{n} \le 0.1 \quad \Rightarrow \quad n \ge 64000.
\]  
因此,取 \(n = 64000\)。

3.
由于 \(X_i \in [0, 20]\),则 \(b - a = 20\)。Chernoff–Hoeffding 不等式给出:  
\[
Pr[|\bar X - \mu| \ge \varepsilon] \le 2 \exp\left(-\frac{2n \varepsilon^2}{(b-a)^2}\right) = 2 \exp\left(-\frac{2n \varepsilon^2}{400}\right) = 2 \exp\left(-\frac{n \varepsilon^2}{200}\right).
\]  
令 \(\varepsilon = 1\),要求 \(Pr[|\bar X - \mu| \ge 1] \le 0.5\),即  
\[
2 \exp\left(-\frac{n}{200}\right) \le 0.5 \quad \Rightarrow \quad \exp\left(-\frac{n}{200}\right) \le 0.25 \quad \Rightarrow \quad -\frac{n}{200} \le \ln(0.25) = -\ln(4).
\]  
因此,  
\[
n \ge 200 \ln(4) \approx 200 \times 1.3863 = 277.26.
\]  
取整数 \(n = 278\)。

4. 
令 \(\varepsilon = 0.1\),要求 \(Pr[|\bar X - \mu| \ge 0.1] \le 0.1\),即  
\[
2 \exp\left(-\frac{n \cdot (0.1)^2}{200}\right) \le 0.1 \quad \Rightarrow \quad 2 \exp\left(-\frac{n}{20000}\right) \le 0.1 \quad \Rightarrow \quad \exp\left(-\frac{n}{20000}\right) \le 0.05.
\]  
因此,  
\[
-\frac{n}{20000} \le \ln(0.05) = -\ln(20) \quad \Rightarrow \quad n \ge 20000 \ln(20) \approx 20000 \times 2.9957 = 59914.
\]  
取整数 \(n = 59915\)。

\end{solution}
\end{exercise}

\begin{exercise}{Problem 2.6}
\begin{problem}
Consider two random variables \( C \) and \( T \) describing how many coffees and teas I will buy in the coming week; clearly neither can be smaller than 0. Based on personal experience, I know the following summary statistics about my coffee and tea buying habits:  \( \mathbb{E}[C] = 3 \) and \( \text{Var}[C] = 1 \)  , \( \mathbb{E}[T] = 2 \) and \( \text{Var}[T] = 5 \)

1.Use Markov's inequality to upper-bound the probability that I buy 4 or more coffees, and the same for teas:  
   \( \Pr[C \geq 4] \) and \( \Pr[T \geq 4] \)

2. Use Chebyshev's inequality to upper-bound the probability that I buy 4 or more coffees, and the same for teas:  
   \( \Pr[C \geq 4] \)  and \( \Pr[T \geq 4] \)

\end{problem}
\begin{solution}

1. 使用马尔可夫不等式:  
马尔可夫不等式指出,对于非负随机变量 \(X\) 和常数 \(a > 0\),有  
\[
\Pr[X \geq a] \leq \frac{\mathbb{E}[X]}{a}.
\]  
对于咖啡 \(C\),有 \(\mathbb{E}[C] = 3\),取 \(a = 4\),得  
\[
\Pr[C \geq 4] \leq \frac{3}{4}.
\]  
对于茶 \(T\),有 \(\mathbb{E}[T] = 2\),取 \(a = 4\),得  
\[
\Pr[T \geq 4] \leq \frac{2}{4} = \frac{1}{2}.
\]  
因此,马尔可夫上界为 \(\Pr[C \geq 4] \leq \frac{3}{4}\) 和 \(\Pr[T \geq 4] \leq \frac{1}{2}\).

2. 使用切比雪夫不等式:  
切比雪夫不等式指出,对于随机变量 \(X\) 和常数 \(k > 0\),有  
\[
\Pr[|X - \mathbb{E}[X]| \geq k] \leq \frac{\text{Var}[X]}{k^2}.
\]  
对于咖啡 \(C\),有 \(\mathbb{E}[C] = 3\),\(\text{Var}[C] = 1\)。注意到  
\[
\Pr[C \geq 4] = \Pr[C - 3 \geq 1] \leq \Pr[|C - 3| \geq 1] \leq \frac{1}{1^2} = 1.
\]  
对于茶 \(T\),有 \(\mathbb{E}[T] = 2\),\(\text{Var}[T] = 5\)。注意到  
\[
\Pr[T \geq 4] = \Pr[T - 2 \geq 2] \leq \Pr[|T - 2| \geq 2] \leq \frac{5}{2^2} = \frac{5}{4}.
\]  
因此,切比雪夫上界为 \(\Pr[C \geq 4] \leq 1\) 和 \(\Pr[T \geq 4] \leq \frac{5}{4}\).  


\end{solution}
\end{exercise}

\begin{exercise}{EX\_Problem 1 }
Let \( X \) be a discrete random variable, with \( X \in [-1, 1] \) and \( \mathbb{E}[X] = 0 \). If \( t \in [0, 1] \), prove that  
\[
\mathbb{E}[e^{tX}] \le 1 + t^2 \text{Var}[X] \le e^{t^2 \text{Var}[X]}.
\]

\begin{proof}
由于 \(X \in [-1, 1]\) 且 \(\mathbb{E}[X] = 0\),考虑函数 \(e^{tx}\) 在区间 \([-1, 1]\) 上的性质。

对于任意 \(x \in [-1, 1]\),函数 \(e^{tx}\) 是凸函数,因此它在该区间上位于连接点 \((-1, e^{-t})\) 和 \((1, e^{t})\) 的线段下方,即:
\[
e^{tx} \le \frac{1 - x}{2} e^{-t} + \frac{1 + x}{2} e^{t} \quad \text{对所有 } x \in [-1,1]
\]

对上述不等式两边取期望(关于 \(X\)),并利用 \(\mathbb{E}[X] = 0\):
\[
\mathbb{E}[e^{tX}] \le \mathbb{E}\left[ \frac{e^{-t} + e^{t}}{2} + X \cdot \frac{e^{t} - e^{-t}}{2} \right] = \frac{e^{-t} + e^{t}}{2} + \frac{e^{t} - e^{-t}}{2} \mathbb{E}[X] = \frac{e^{-t} + e^{t}}{2}
\]

已知 \(\cosh(t) = \frac{e^{t} + e^{-t}}{2}\),所以:
\[
\mathbb{E}[e^{tX}] \le \cosh(t)
\]

利用不等式 \(\cosh(t) \le 1 + \frac{t^2}{2}\)(当 \(t \in [0,1]\) 时成立),以及 \(\text{Var}[X] = \mathbb{E}[X^2] \le 1\)(因为 \(X^2 \le 1\)),可得:
\[
\mathbb{E}[e^{tX}] \le 1 + \frac{t^2}{2} \le 1 + t^2 \text{Var}[X]
\]

最后,利用不等式 \(1 + y \le e^{y}\)(对所有实数 \(y\) 成立),令 \(y = t^2 \text{Var}[X]\),得:
\[
1 + t^2 \text{Var}[X] \le e^{t^2 \text{Var}[X]}
\]

综上,不等式链得证。
\end{proof}
\end{exercise}

\begin{exercise}{EX\_Problem 2 }
Let \( X_1, \dots, X_n \) be discrete, identical and independent random variables with \( \mathbb{E}[X_i] = 0 \), \( \text{Var}[X_i] = \sigma^2 \), and \( X_i \in [-1, 1] \). Let \( \bar{X} = \frac{1}{n} \sum_{i=1}^n X_i \). Prove that  
\[
\Pr[|\bar{X}| \ge \lambda \sigma] \le 2e^{-\lambda^2 n / 4}
\]  
for any \( 0 \le \lambda \le 2\sigma \).

\begin{proof}
  令 \(s = \frac{\lambda}{2\sigma}\),由于 \(0 \le \lambda \le 2\sigma\),有 \(s \in [0,1]\),满足问题1的条件。

对于任意 \(s > 0\),根据马尔可夫不等式:
\[
\Pr[\bar{X} \ge \lambda \sigma] = \Pr\left[e^{s n \bar{X}} \ge e^{s n \lambda \sigma}\right] \le \frac{\mathbb{E}[e^{s n \bar{X}}]}{e^{s n \lambda \sigma}}
\]

由于 \(X_i\) 独立同分布,且 \(\bar{X} = \frac{1}{n} \sum_{i=1}^n X_i\),有:
\[
\mathbb{E}[e^{s n \bar{X}}] = \mathbb{E}[e^{s \sum_{i=1}^n X_i}] = \prod_{i=1}^n \mathbb{E}[e^{s X_i}] = (\mathbb{E}[e^{s X_1}])^n
\]

应用问题1的结论,有 \(\mathbb{E}[e^{s X_1}] \le e^{s^2 \text{Var}[X_1]} = e^{s^2 \sigma^2}\),因此:
\[
\Pr[\bar{X} \ge \lambda \sigma] \le \frac{(e^{s^2 \sigma^2})^n}{e^{s n \lambda \sigma}} = e^{n s^2 \sigma^2 - s n \lambda \sigma}
\]

令 \(s = \frac{\lambda}{2\sigma}\),代入上式:
\[
\Pr[\bar{X} \ge \lambda \sigma] \le e^{n \cdot (\frac{\lambda}{2\sigma})^2 \sigma^2 - \frac{\lambda}{2\sigma} \cdot n \lambda \sigma} = e^{n \cdot \frac{\lambda^2}{4} - n \cdot \frac{\lambda^2}{2}} = e^{-\frac{\lambda^2 n}{4}}
\]

同理,对于 \(\Pr[\bar{X} \le -\lambda \sigma]\),可以得到相同的上界。

因此,结合两侧:
\[
\Pr[|\bar{X}| \ge \lambda \sigma] = \Pr[\bar{X} \ge \lambda \sigma] + \Pr[\bar{X} \le -\lambda \sigma] \le 2e^{-\frac{\lambda^2 n}{4}}
\]

\end{proof}
\end{exercise}
\end{document}