\documentclass[12pt,a4paper]{article}
\usepackage[UTF8]{ctex}         % 中文支持
\usepackage{amsmath, amssymb}   % 数学公式
\usepackage{geometry}           % 页面设置
\usepackage{hyperref}           % 超链接
\usepackage{graphicx}           % 插图
\usepackage{enumerate}          % 枚举
\usepackage{booktabs}           % 表格
\usepackage{lmodern}            % 现代字体
\usepackage{microtype}          % 微排版优化
\usepackage{xcolor}             % 颜色支持
\usepackage[most]{tcolorbox}   % 彩色盒子
\geometry{left=2cm, right=2cm, top=2cm, bottom=2.5cm}

\definecolor{ProblemFrame}{HTML}{2F5597}
\definecolor{ProblemBack}{HTML}{EEF2FB}
\definecolor{SolutionColor}{HTML}{007060}

\hypersetup{
	colorlinks=true,
	linkcolor=ProblemFrame,
	citecolor=SolutionColor,
	urlcolor=ProblemFrame
}

% --- 大标题样式调整 ---
\makeatletter
\def\@maketitle{%
  \newpage
  \null
  \vspace{-2em} % 减少顶部空白
  \begin{center}%
  \let \footnote \thanks
    {\LARGE \@title \par}%
    \vskip 1.5em%
    {\large
      \lineskip .5em%
      \begin{tabular}[t]{c}%
        \@author
      \end{tabular}\par}%
    \vskip 1em%
    {\large \@date}%
  \end{center}%
  \par
  \vskip 1.5em}
\makeatother


% --- 新的题解环境 ---
\newtcolorbox{exercisebox}[2][]{
    enhanced,
    breakable,
    sharp corners,
    colback=white,
    colframe=black!15!white,
    coltitle=white,
    fonttitle=\bfseries\large,
    attach boxed title to top left={xshift=1.5mm, yshift*=-\tcboxedtitleheight/2},
    boxed title style={colback=ProblemFrame, sharp corners},
    title={#2},
    before skip=18pt,
    after skip=18pt,
    #1
}

\newtcolorbox{problemnobox}[1][]{
    enhanced,
    sharp corners,
    boxrule=0.8pt,
    colframe=black!25!white,
    colback=ProblemBack,
    breakable,
    top=4pt, bottom=4pt, left=5pt, right=5pt,
    #1
}

\newenvironment{exercise}[2][]
  {\begin{exercisebox}[#1]{#2}\ignorespaces}
  {\end{exercisebox}}

\newenvironment{problem}
  {\begin{problemnobox}\ignorespaces}
  {\end{problemnobox}}

\newenvironment{solution}
  {\par\noindent\textbf{\color{SolutionColor}解:}\ignorespaces}
  {}

\newenvironment{proof}
  {\par\noindent\textbf{\color{SolutionColor}证明:}\ignorespaces}
  {}

\setlength{\parskip}{0.85em}
\setlength{\parindent}{2em}

\title{数据科学的数学基础 \\ 第{{4}}次理论作业 }
\author{ 陈万祺 \quad 3220102895}
\date{\today}

\begin{document}

\maketitle

\begin{exercise}{Problem 1}
\begin{problem}
Show that in the space of \(\mathbb{R}^n\), for any \(1 \leq p_1 < p_2 \leq +\infty\), there exist \(c_1 > 0\) and \(c_2 > 0\), which depend on \(n\), \(p_1\) and \(p_2\), such that for any \(x \in \mathbb{R}^n\), it holds
\[
c_1\|x\|_{p_1} \leq \|x\|_{p_2} \leq c_2\|x\|_{p_1}.
\]
\end{problem}
\begin{solution}

首先,考虑 \(p_2 < +\infty\) 的情况。对于任意 \(x \in \mathbb{R}^n\),我们有

\[
\|x\|_{p_2}^{p_2} = \sum_{i=1}^n |x_i|^{p_2}.
\]

由于 \(p_1 < p_2\),利用 Hölder 不等式。设 \(q = \frac{p_2}{p_1} > 1\),则 \(q' = \frac{q}{q-1} = \frac{p_2}{p_2 - p_1}\)。根据 Hölder 不等式,

\[
\sum_{i=1}^n |x_i|^{p_2} = \sum_{i=1}^n (|x_i|^{p_1})^{q} \leq \left( \sum_{i=1}^n |x_i|^{p_1} \right)^{q} \cdot \left( \sum_{i=1}^n 1^{q'} \right)^{1/q'} = \|x\|_{p_1}^{p_2} \cdot n^{1 - p_1/p_2}.
\]

因此,

\[
\|x\|_{p_2} \leq n^{(1 - p_1/p_2)/p_2} \|x\|_{p_1} = n^{1/p_1 - 1/p_2} \|x\|_{p_1}.
\]

所以,我们可以取 \(c_2 = n^{1/p_1 - 1/p_2}\)。

接下来,考虑 \(p_2 = +\infty\) 的情况。此时,

\[
\|x\|_{\infty} = \max_{1 \leq i \leq n} |x_i|.
\]

对于任意 \(x \in \mathbb{R}^n\),我们有

\[
|x_i| \leq \left( \sum_{j=1}^n |x_j|^{p_1} \right)^{1/p_1} = \|x\|_{p_1},
\]

对于所有 \(i = 1, \dots, n\)。因此,

\[
\|x\|_{\infty} \leq \|x\|_{p_1}.
\]

所以,我们可以取 \(c_2 = 1\)。

接下来证明不等式的左侧部分。

首先,考虑 \(p_2 < +\infty\) 的情况。对于任意 \(x \in \mathbb{R}^n\),我们有

\[
\|x\|_{p_1}^{p_1} = \sum_{i=1}^n |x_i|^{p_1}.
\]

由于 \(p_1 < p_2\),根据 Jensen 不等式,

\[
\left( \frac{1}{n} \sum_{i=1}^n |x_i|^{p_1} \right)^{p_2/p_1} \leq \frac{1}{n} \sum_{i=1}^n (|x_i|^{p_1})^{p_2/p_1} = \frac{1}{n} \sum_{i=1}^n |x_i|^{p_2}.
\]

因此,

\[
\left( \frac{\|x\|_{p_1}^{p_1}}{n} \right)^{p_2/p_1} \leq \frac{\|x\|_{p_2}^{p_2}}{n}.
\]

整理得

\[
\|x\|_{p_1}^{p_2} \leq n^{p_2/p_1 - 1} \|x\|_{p_2}^{p_2}.
\]

因此,

\[
\|x\|_{p_1} \leq n^{1/p_1 - 1/p_2} \|x\|_{p_2}.
\]

所以,我们可以取 \(c_1 = n^{1/p_2 - 1/p_1}\)。

接下来,考虑 \(p_2 = +\infty\) 的情况。此时,

\[
\|x\|_{\infty} = \max_{1 \leq i \leq n} |x_i|.
\]

对于任意 \(x \in \mathbb{R}^n\),我们有

\[
\|x\|_{p_1}^{p_1} = \sum_{i=1}^n |x_i|^{p_1} \leq n \cdot (\max_{1 \leq i \leq n} |x_i|)^{p_1} = n \|x\|_{\infty}^{p_1}.
\]

因此,

\[
\|x\|_{p_1} \leq n^{1/p_1} \|x\|_{\infty}.
\]

所以,我们可以取 \(c_1 = n^{-1/p_1}\)。

综上所述,对于任意 \(1 \leq p_1 < p_2 \leq +\infty\),存在依赖于 \(n\)、\(p_1\) 和 \(p_2\) 的常数 \(c_1 > 0\) 和 \(c_2 > 0\),使得对于任意 \(x \in \mathbb{R}^n\),有

\[
c_1\|x\|_{p_1} \leq \|x\|_{p_2} \leq c_2\|x\|_{p_1}.
\]

具体地,当 \(p_2 < +\infty\) 时,可以取 \(c_1 = n^{1/p_2 - 1/p_1}\) 和 \(c_2 = n^{1/p_1 - 1/p_2}\);当 \(p_2 = +\infty\) 时,可以取 \(c_1 = n^{-1/p_1}\) 和 \(c_2 = 1\)。

\end{solution}
\end{exercise}

\begin{exercise}{Problem 2}
\begin{problem}
Prove or disprove: for any \(x \in \mathbb{R}^n\), it holds
\[
\|x\|_1 \|x\|_\infty \leq \frac{1 + \sqrt{n}}{2} \|x\|_2^2.
\]
\end{problem}
\begin{solution}

不妨假设 \(\|x\|_\infty = 1\)(若 \(x = 0\) 则不等式显然成立)。令 \(y_i = |x_i|\),则 \(y_i \in [0, 1]\) 且 \(\max_i y_i = 1\)。此时需证:
\[
\sum_{i=1}^n y_i \leq \frac{1 + \sqrt{n}}{2} \sum_{i=1}^n y_i^2.
\]

等价地:
\[
(1 + \sqrt{n}) \sum_{i=1}^n y_i^2 - 2 \sum_{i=1}^n y_i \geq 0.
\]

考虑函数 \(f(t) = 2t - (1 + \sqrt{n})t^2\),其中 \(t \in [0, 1]\)。这是开口向下的二次函数,顶点在 \(t = \frac{1}{1 + \sqrt{n}}\) 处,最大值为 \(\frac{1}{1 + \sqrt{n}}\)。因此对任意 \(t \in [0, 1]\):
\[
2t - (1 + \sqrt{n})t^2 \leq \frac{1}{1 + \sqrt{n}}.
\]

不失一般性,设 \(y_1 = 1\)。则:
- 当 \(i = 1\) 时:\(2 - (1 + \sqrt{n}) = 1 - \sqrt{n}\)
- 当 \(i = 2, \dots, n\) 时:\(2y_i - (1 + \sqrt{n})y_i^2 \leq \frac{1}{1 + \sqrt{n}}\)

对所有 \(i\) 求和:
\[
\sum_{i=1}^n \left[ 2y_i - (1 + \sqrt{n})y_i^2 \right] \leq (1 - \sqrt{n}) + (n - 1) \cdot \frac{1}{1 + \sqrt{n}}.
\]

计算右边:
\[
(1 - \sqrt{n}) + \frac{n - 1}{1 + \sqrt{n}} = (1 - \sqrt{n}) + \frac{(\sqrt{n} - 1)(\sqrt{n} + 1)}{1 + \sqrt{n}} = (1 - \sqrt{n}) + (\sqrt{n} - 1) = 0.
\]

因此:
\[
\sum_{i=1}^n \left[ 2y_i - (1 + \sqrt{n})y_i^2 \right] \leq 0,
\]
即:
\[
2 \sum y_i \leq (1 + \sqrt{n}) \sum y_i^2 \quad \Longleftrightarrow \quad \sum y_i \leq \frac{1 + \sqrt{n}}{2} \sum y_i^2.
\]

由于在假设 \(\|x\|_\infty = 1\) 下不等式成立,且不等式是齐次的,故对任意 \(x \in \mathbb{R}^n\) 均成立。

\vspace{0.2cm}
\noindent 因此,原不等式成立。

\end{solution}
\end{exercise}


\end{document}