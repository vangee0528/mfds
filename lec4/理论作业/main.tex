\documentclass[12pt,a4paper]{article}
\usepackage[UTF8]{ctex}         % 中文支持
\usepackage{amsmath, amssymb}   % 数学公式
\usepackage{geometry}           % 页面设置
\usepackage{hyperref}           % 超链接
\usepackage{graphicx}           % 插图
\usepackage{enumerate}          % 枚举
\usepackage{booktabs}           % 表格
\usepackage{lmodern}            % 现代字体
\usepackage{microtype}          % 微排版优化
\usepackage{xcolor}             % 颜色支持
\usepackage[most]{tcolorbox}   % 彩色盒子
\geometry{left=2cm, right=2cm, top=2cm, bottom=2.5cm}

\definecolor{ProblemFrame}{HTML}{2F5597}
\definecolor{ProblemBack}{HTML}{EEF2FB}
\definecolor{SolutionColor}{HTML}{007060}

\hypersetup{
	colorlinks=true,
	linkcolor=ProblemFrame,
	citecolor=SolutionColor,
	urlcolor=ProblemFrame
}

% --- 大标题样式调整 ---
\makeatletter
\def\@maketitle{%
  \newpage
  \null
  \vspace{-2em} % 减少顶部空白
  \begin{center}%
  \let \footnote \thanks
    {\LARGE \@title \par}%
    \vskip 1.5em%
    {\large
      \lineskip .5em%
      \begin{tabular}[t]{c}%
        \@author
      \end{tabular}\par}%
    \vskip 1em%
    {\large \@date}%
  \end{center}%
  \par
  \vskip 1.5em}
\makeatother


% --- 新的题解环境 ---
\newtcolorbox{exercisebox}[2][]{
    enhanced,
    breakable,
    sharp corners,
    colback=white,
    colframe=black!15!white,
    coltitle=white,
    fonttitle=\bfseries\large,
    attach boxed title to top left={xshift=1.5mm, yshift*=-\tcboxedtitleheight/2},
    boxed title style={colback=ProblemFrame, sharp corners},
    title={#2},
    before skip=18pt,
    after skip=18pt,
    #1
}

\newtcolorbox{problemnobox}[1][]{
    enhanced,
    sharp corners,
    boxrule=0.8pt,
    colframe=black!25!white,
    colback=ProblemBack,
    breakable,
    top=4pt, bottom=4pt, left=5pt, right=5pt,
    #1
}

\newenvironment{exercise}[2][]
  {\begin{exercisebox}[#1]{#2}\ignorespaces}
  {\end{exercisebox}}

\newenvironment{problem}
  {\begin{problemnobox}\ignorespaces}
  {\end{problemnobox}}

\newenvironment{solution}
  {\par\noindent\textbf{\color{SolutionColor}解:}\ignorespaces}
  {}

\newenvironment{proof}
  {\par\noindent\textbf{\color{SolutionColor}证明:}\ignorespaces}
  {}

\setlength{\parskip}{0.85em}
\setlength{\parindent}{2em}

\title{数据科学的数学基础 \\ 第{{3}}次理论作业 }
\author{ 陈万祺 \quad 3220102895}
\date{\today}

\begin{document}

\maketitle

\begin{exercise}{Problem 2.7}
\begin{problem}
The average score on a test is 82 with a standard deviation of 4 percentage points.  
All tests have scores between 0 and 100.

1. Using Chebyshev's inequality, what percentage of the tests have a grade of at least 70 and at most 94?  

2. Using Markov's inequality, what is the highest percentage of tests which could have a score less than 60?
\end{problem}
\begin{solution}

1. 均值 $\mu = 82$,标准差 $\sigma = 4$。  
区间 $[70, 94]$ 对称于均值,偏差为 $|82 - 70| = 12$,因此 $k\sigma = 12$,$k = 12 / 4 = 3$。  
由切比雪夫不等式
\[P(|X - \mu| \leq k\sigma) \geq 1 - \frac{1}{k^2}\]
所以
\[P(70 \leq X \leq 94) \geq 1 - \frac{1}{3^2} = \frac{8}{9} \approx 88.89\%\]

因此,至少约 $88.89\%$ 的测试分数在70和94之间。

2. 定义 $Y = 100 - X$,则 $Y$ 非负,且 $E[Y] = 100 - 82 = 18$。  
$P(X < 60) = P(Y > 40)$。  
由马尔可夫不等式
\[P(Y \geq a) \leq \frac{E[Y]}{a}\]
因此
\[P(Y > 40) \leq \frac{18}{40} = 0.45\]
所以分数小于60的最高百分比为 $45\%$。
\end{solution}
\end{exercise}

\begin{exercise}{Problem 2.8}
\begin{problem}
Consider a random variable \( X \) with expected values \( E[X] = 7 \) and variance \( \text{Var}[X] = 2 \). We would like to upper-bound the probability \( \Pr[X < 5] \).

1. Which bound can and cannot be used with what we know about \( X \) (Markov, Chebyshev, or Chernoff-Hoeffding), and why?  

2. Using that bound, calculate an upper bound for \( \Pr[X < 5] \).  

3. Describe a probability distribution for \( X \) where the other two bounds are definitely not applicable.
\end{problem}
\begin{solution}
1. 边界使用情况:  
马尔可夫不等式:无法使用,因为马尔可夫不等式要求随机变量非负,而 \( X \) 可能取负值,且无法直接用于 \( \Pr[X < 5] \) 的上界。 

切比雪夫不等式:可以使用,因为我们知道方差,且切比雪夫不等式适用于任何具有有限方差的随机变量。  

切尔诺夫-霍夫丁不等式:无法使用,因为该不等式通常适用于有界独立随机变量的和或平均,而这里只有一个随机变量,且不知道其分布或有界性。

2.
由切比雪夫不等式:
\[ \Pr[|X - \mu| \geq k] \leq \frac{\text{Var}[X]}{k^2} \]

这里 $ \mu = 7 $,且
\[ \Pr[X < 5] = \Pr[X - 7 < -2] \leq \Pr[|X - 7| \geq 2] \]
因为事件 $ X - 7 < -2 $ 是 $ |X - 7| \geq 2 $ 的子集。
因此
\[ \Pr[X < 5] \leq \Pr[|X - 7| \geq 2] \leq \frac{2}{2^2} = 0.5 \]
因此,$ \Pr[X < 5] $ 的上界为 0.5。

3. 描述一个概率分布,其中其他两个边界肯定不适用:  
考虑 $ X $ 服从正态分布 $ N(7, 2) $,即均值 7,方差 2。  
马尔可夫不等式不适用,因为 $ X $ 可能取负值(例如,正态分布有负值概率),且马尔可夫不等式要求非负。  
切尔诺夫-霍夫丁不等式不适用,因为该不等式要求随机变量有界,而正态分布无界。
\end{solution}
\end{exercise}

\begin{exercise}{Problem 2.9}
\begin{problem}
Consider \( n \) iid random variables \( X_1, X_2, \ldots, X_n \) with expected value \( E[X_i] = 20 \) and variance \( \text{Var}[X_i] = 2 \). Assume we also know that each \( X_i \) must satisfy \( 15 \leq X_i \leq 22 \). We now want to analyze the random variable of their average \( \bar{X} = \frac{1}{n} \sum_{i=1}^n X_i \).  
Assume first that \( n = 20 \) (the number of random variables).  

1. Use the Chebyshev inequality to upper-bound \( \Pr[\bar{X} > 21] \).  

2. Use the Chernoff-Hoeffding inequality to upper-bound \( \Pr[\bar{X} > 21] \).  
Now assume first that \( n = 200 \) (the number of random variables).  

3. Use the Chebyshev inequality to upper-bound \( \Pr[\bar{X} > 21] \).  

4. Use the Chernoff-Hoeffding inequality to upper-bound \( \Pr[\bar{X} > 21] \).

\end{problem}
\begin{solution}
1. 对于平均 $ \bar{X} $,有 $ E[\bar{X}] = 20 $,$ \text{Var}[\bar{X}] = \frac{\text{Var}[X_i]}{n} = \frac{2}{n} $。  
由于 $ X_i \in [15, 22] $,所以 $ \bar{X} \in [15, 22] $。  
令 $ \mu = 20 $,$ t = 1 $(因为 $ \bar{X} > 21 $ 等价于 $ \bar{X} - \mu > 1 $)。

当 $ n = 20 $:  
1. 切比雪夫不等式:
\[ \Pr[\bar{X} > 21] = \Pr[\bar{X} - 20 > 1] \leq \Pr[|\bar{X} - 20| \geq 1] \leq \frac{\text{Var}[\bar{X}]}{1^2} = 0.1 \]
所以上界为 0.1。

2. 切尔诺夫-霍夫丁不等式:  
由于 $ X_i \in [15, 22] $,范围 $ b - a = 22 - 15 = 7 $。  
切尔诺夫-霍夫丁给出 $ \Pr[\bar{X} - \mu \geq t] \leq \exp\left( -\frac{2n t^2}{(b-a)^2} \right) $。  
所以 $ \Pr[\bar{X} > 21] \leq \exp\left( -\frac{2 \times 20 \times 1^2}{7^2} \right) = \exp\left( -\frac{40}{49} \right) \approx \exp(-0.8163) \approx 0.441 $。  
所以上界约为 0.441。

当 $ n = 200 $:  

3. 切比雪夫不等式:
\[ \Pr[\bar{X} > 21] \leq \frac{\text{Var}[\bar{X}]}{1^2} = 0.01 \]
所以上界为 0.01。

4. 切尔诺夫-霍夫丁不等式:  
$ \Pr[\bar{X} > 21] \leq \exp\left( -\frac{2 \times 200 \times 1^2}{7^2} \right) = \exp\left( -\frac{400}{49} \right) \approx \exp(-8.163) \approx 0.00028 $。  
所以上界约为 0.00028。
\end{solution}
\end{exercise}

\begin{exercise}{EXProblem 1}
\begin{problem}
Given the error tolerance $\varepsilon = 0.01$, at least how many samples do we need to approximate all quantiles of a distribution so that the probability of failure $\delta = 0.05$?
\end{problem}
\begin{solution}
为了使用经验分布函数一致逼近真实分布函数的所有分位数,我们使用Dvoretzky-Kiefer-Wolfowitz (DKW) 不等式:

\[
P\left(\sup_x |F_n(x) - F(x)| > \varepsilon\right) \leq 2\exp(-2n\varepsilon^2)
\]

其中 $F_n(x)$ 是经验分布函数,$F(x)$ 是真实分布函数,$n$ 是样本数量。

要确保失败概率不超过 $\delta = 0.05$,我们需要:

\[
2\exp(-2n\varepsilon^2) \leq \delta
\]

代入 $\varepsilon = 0.01$ 和 $\delta = 0.05$:

\[
2\exp(-2n(0.01)^2) \leq 0.05
\]

\[
-0.0002n \leq \ln(0.025)
\]

由于 $\ln(0.025) \approx -3.688879$,所以:

\[
n \geq \frac{3.688879}{0.0002} = 18444.395
\]

因此,至少需要 $18445$ 个样本才能以 $95\%$ 的置信水平确保所有分位数的估计误差不超过 $0.01$。
\end{solution}
\end{exercise}
\end{document}